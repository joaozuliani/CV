% TEMPLATE DE CV

%----------------------------------------------------------------------------------------
%	DOCUMENT DEFINITION
%----------------------------------------------------------------------------------------

% article class because we want to fully customize the page and not use a cv template
\documentclass[a4paper,12pt]{article}

%----------------------------------------------------------------------------------------
%	FONT
%----------------------------------------------------------------------------------------

% % fontspec allows you to use TTF/OTF fonts directly
% \usepackage{fontspec}
% \defaultfontfeatures{Ligatures=TeX}

% % modified for ShareLaTeX use
% \setmainfont[
% SmallCapsFont = Fontin-SmallCaps.otf,
% BoldFont = Fontin-Bold.otf,
% ItalicFont = Fontin-Italic.otf
% ]
% {Fontin.otf}

%----------------------------------------------------------------------------------------
%	PACKAGES
%----------------------------------------------------------------------------------------
\usepackage{url}
\usepackage{parskip} 	
\usepackage{blindtext}
\usepackage{multicol}
\usepackage[portuguese]{babel}

%other packages for formatting
\RequirePackage{color}
\RequirePackage{graphicx}
\usepackage[usenames,dvipsnames]{xcolor}
\usepackage[scale=0.9]{geometry}

%tabularx environment
\usepackage{tabularx}

%for lists within experience section
\usepackage{enumitem}

% centered version of 'X' col. type
\newcolumntype{C}{>{\centering\arraybackslash}X} 

%to prevent spillover of tabular into next pages
\usepackage{supertabular}
\usepackage{tabularx}
\newlength{\fullcollw}
\setlength{\fullcollw}{0.47\textwidth}

%custom \section
\usepackage{titlesec}				
\usepackage{multicol}
\usepackage{multirow}

%CV Sections inspired by: 
%http://stefano.italians.nl/archives/26
\titleformat{\section}{\Large\scshape\raggedright}{}{0em}{}[\titlerule]
\titlespacing{\section}{0pt}{10pt}{10pt}

%for publications
\usepackage[style=authoryear,sorting=ynt, maxbibnames=2]{biblatex}

%Setup hyperref package, and colours for links
\usepackage[unicode, draft=false]{hyperref}
\definecolor{linkcolour}{rgb}{0,0.2,0.6}
\hypersetup{colorlinks,breaklinks,urlcolor=linkcolour,linkcolor=linkcolour}
\addbibresource{citations.bib}
\setlength\bibitemsep{1em}

%for social icons
\usepackage{fontawesome5}

%debug page outer frames
%\usepackage{showframe}

%----------------------------------------------------------------------------------------
%	BEGIN DOCUMENT
%----------------------------------------------------------------------------------------
\begin{document}

% non-numbered pages
\pagestyle{empty} 

%----------------------------------------------------------------------------------------
%	TITLE
%----------------------------------------------------------------------------------------

% \begin{tabularx}{\linewidth}{ @{}X X@{} }
% \huge{Your Name}\vspace{2pt} & \hfill \emoji{incoming-envelope} email@email.com \\
% \raisebox{-0.05\height}\faGithub\ username \ | \
% \raisebox{-0.00\height}\faLinkedin\ username \ | \ \raisebox{-0.05\height}\faGlobe \ mysite.com  & \hfill \emoji{calling} number
% \end{tabularx}

\begin{tabularx}{\linewidth}{@{} C @{}}
\Huge{João Lucas Ferreira Zuliani} \\[7.5pt]
\href{https://github.com/joaozuliani}{\raisebox{-0.05\height}\faGithub\ joaozuliani} \ $|$ \ 
\href{https://www.linkedin.com/in/joão-lucas-zuliani/}{\raisebox{-0.05\height}\faLinkedin\ joão-lucas-zuliani} \ $|$ \ 
\href{mailto:joao.lucas.zuliani@usp.br}{\raisebox{-0.05\height}\faEnvelope \ joao.lucas.zuliani@usp.br} \ $|$ \ 
\href{tel:+5519982391596}{\raisebox{-0.05\height}\faMobile \ +55 19 98239-1596} \\
\textcolor{blue}{\raisebox{-0.05\height}\faGlobe \ São Carlos, SP} \\
\end{tabularx}

%----------------------------------------------------------------------------------------
%	EDUCATION
%----------------------------------------------------------------------------------------
\section{Formação}
\begin{tabularx}{\linewidth}{ @{}l X@{} }
\textbf{Bacharelado em Engenharia Mecatrônica}, Universidade de São Paulo & \hfill Fev 2019 - Dez 2023\\[3.75pt]
\end{tabularx}

%----------------------------------------------------------------------------------------
% EXPERIENCE SECTIONS
%----------------------------------------------------------------------------------------

%Interests/ Keywords/ Summary
%\section{Interesses}
%Estudante de Engenhariua Mecatrôncia.

%Experience
\section{Experiência}

\begin{tabularx}{\linewidth}{ @{}l r@{} }
\textbf{Coordenador Geral, Secretaria Acadêmica da Engenharia Mecatrônica} & \hfill Set 2021 - Out 2022 \\[3.75pt]
\multicolumn{2}{@{}X@{}}{
\begin{minipage}[t]{\linewidth}
    \begin{itemize}[nosep,after=\strut, leftmargin=1em, itemsep=3pt]
        \item[--] Atuação como liderança e organização da secretaria, coordenando os gestores e membros.
        \item[--] Desenvolvimento de soft skills para comunicação com professores, outras secretarias, instituições e alunos.
        \item[--] Gerenciamento da parte de comunicação da secretaria.
        \item[--] Atuação direta em eventos de cunho social e cultural realizado pelos alunos da engenharia mecatrônica.
    \end{itemize}
    \end{minipage}
}
\end{tabularx}

\begin{tabularx}{\linewidth}{ @{}l r@{} }
\textbf{Gestor, Secretaria Acadêmica da Engenharia Mecatrônica} & \hfill Jan 2020 - Set 2022 \\[3.75pt]
\multicolumn{2}{@{}X@{}}{
\begin{minipage}[t]{\linewidth}
    \begin{itemize}[nosep,after=\strut, leftmargin=1em, itemsep=3pt]
        \item[--] Liderança do núcleo de esportes da secretaria.
        \item[--] Organização, planejamento e realização de eventos de cunho esportivo para os alunos da universidade.
    \end{itemize}
    \end{minipage}
}
\end{tabularx}

\begin{tabularx}{\linewidth}{ @{}l r@{} }
\textbf{Membro, Secretaria Acadêmica da Engenharia Mecatrônica} & \hfill Mar 2019 - Jun 2022 \\[3.75pt]
\multicolumn{2}{@{}X@{}}{
\begin{minipage}[t]{\linewidth}
    \begin{itemize}[nosep,after=\strut, leftmargin=1em, itemsep=3pt]
        \item[--] Desenvolvimento e venda de produtos para os estudantes da engenharia mecatrônica do campus.
        \item[--] Contato com fornecedores e empresas que confeccionam produtos personalizados.
    \end{itemize}
    \end{minipage}
}
\end{tabularx}

%Projects
\section{Projetos}

\begin{tabularx}{\linewidth}{ @{}l r@{} }
\textbf{Iniciação Científica, LegRo Group - Laboratório de Robótica, USP} & \hfill Set 2022 - presente\\[3pt]
\multicolumn{2}{@{}X@{}}{
\begin{minipage}[t]{\linewidth}
    \begin{itemize}[nosep,after=\strut, leftmargin=1em, itemsep=3pt]
        \item[--] Aplicação de novas métricas no controle de um exoesqueleto de membros inferiores utilizado na reabilitação de pacientes com lesões.
        \item[--] Estudo sobre a margem de passividade da órtese utilizada.
        \item[--] Utilização de ferramentas e softwares como: \textbf{MATLAB, LaTeX, ROS, GitHub, Notion}.
        \item[--] Estudos sobre \textbf{Robótica, Controle, Desenvolvimento, Programação}.
    \end{itemize}
    \end{minipage}
    }
\end{tabularx}

%----------------------------------------------------------------------------------------
%	PUBLICATIONS
%----------------------------------------------------------------------------------------
%\section{Cursos e Certificados}
%\begin{refsection}[citations.bib]
%\nocite{*}
%\printbibliography[heading=none]
%\end{refsection}

%----------------------------------------------------------------------------------------
%	SKILLS
%----------------------------------------------------------------------------------------
\section{Cursos e Certificados}
\begin{tabularx}{\linewidth}{@{}l X@{}}
\textbf{Python} &  \normalsize{Minicurso realizado pelo Grupo SEMATRON.}\\
\textbf{MATLAB}  &  \normalsize{Minicurso realizado pelo Grupo SEMATRON.}\\

\end{tabularx}

\begin{multicols}{2}
\section{Interesses}
\begin{tabularx}{\linewidth}{@{}l l@{}}
\begin{minipage}[t]{\linewidth}
    \begin{itemize}[nosep,after=\strut, leftmargin=1em, itemsep=3pt]
        \item[--] Conhecer e aprender novas linguagens de programação.
        \item[--] Aprimorar os conhecimentos nas linguagens já conhecidas.
        \item[--] Obter experiências na área de desenvolvimento.
    \end{itemize}
    \end{minipage}
\end{tabularx}
\section{Idiomas}
\begin{tabularx}{\linewidth}{@{}l l@{}}
\textbf{Português} & \normalsize{Fluente.}\\
\textbf{Inglês} & \normalsize{Avançado.}\\
\textbf{Espanhol} & \normalsize{Básico.}\\
\end{tabularx}
\end{multicols}

\vfill
\center{\footnotesize Úlltima atualização: \today}

\end{document}
